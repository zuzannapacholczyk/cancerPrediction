\documentclass{article}
\usepackage{polski}
\usepackage[utf8]{inputenc} 
\usepackage[T1]{fontenc} 
\usepackage{graphicx}
\usepackage[hscale=0.7,vscale=0.8]{geometry}

\title{System wspomagania diagnostyki medycznej
\\ Systemy wspomagania decyzji- projekt}
\author{Zuzanna Pacholczyk, 230709 \\ 
		Dawid Robaczyński, 194860}

\begin{document}

\maketitle

\section{Założenia wstępne}

Celem projektu jest stworzenie aplikacji wspomagającej diagnostykę medyczną w zakresie wykrywania raka piersi. Aplikacja wykorzystuje dane dostarczone przez użytkownika lub wcześniej przygotowane (dane zaczerpnięte z UCI Marchine Learning Repository), dostarczane wraz z aplikacją w celu zbudowania klasyfikatora wykorzystywanego do określenia prawdopodobieństwa występowania raka piersi. Program pozwala użytkownikowi na samodzielne wprowadzenie wartości szeregu atrybutów i obliczenie na ich podstawie prawdopodobieństwa choroby. Wynik zostaje zaprezentowany jako wartość liczbowa wraz z drzewem decyzyjnym, które doprowadziło do uzyskania wyniku.

\section{Technologia}

Projekt został zaimplementowany w języku Java w wersji 8 przy wykorzystaniu framework'a JavaFX w celu utworzenia interfejsu graficznego oraz biblioteki Weka pozwalającej na wykorzystanie algorytmu eksploracji danych (ang. data mining).

\section{Wykorzystane modele i algorytmy}

Dla naszego problemu charakterystyczna jest niewielka liczba atrybutów: 9. Wszystkie z nich to atrybuty numeryczne przyjmujące wartości całkowitoliczbowe z zakresu 1-10, dlatego wybór odpowiedniego modelu chcieliśmy jak najlepiej dopasować do tej charakterystyki.

Pierwszym z analizowanych modeli był Random Forest wykorzystujący kilka drzew decyzyjnych w celu zbudowania klasyfikatora. Opiera się o tzw. bagging (pakowanie do worków) w połączeniu z losowym wyborem ustawień. 

Model ten został jednak przez nas odrzucony ze względu na zbyt skomplikowaną wizualizację, która nie byłaby użyteczna na niedoświadczonego użytkownika oraz zbyt skomplikowane ustawienia dla potrzeb naszego programu.

Model, który wybraliśmy, to J48 wykorzystujący algorytm C4.5, który buduje pojedyncze drzewo decyzyjne. Doskonale sprawdza się w zastosowaniu dla atrybutów numerycznych, pozwala na estetyczną wizualizację powstałego drzewa i daje wiarygodne wyniki.

\section{Działanie aplikacji}

Użytkownik ma możliwość podania ścieżki do pliku z danymi, na podstawie których ma zostać zbudowany klasyfikator. Jeśli informacje te nie zostaną dostarczone, wykorzystane zostaną dane zaczerpnięte z UCI Marchine Learning Repository (\textit{https://archive.ics.uci.edu/ml/datasets/Breast+Cancer+Wisconsin+%28Original%29}) i wcześniej zbudowany klasyfikator.

Następnie użytkownik ma możliwość wprowadzenia wartości atrybutów:
\begin{itemize}
\item grubość warstwy,
\item rozmiar komórki równomierny,
\item kształt komórki równomierny,
\item marginalna przyczepność komórki,
\item tozmiar tkanki nabłonkowej,
\item jądra komórkowe,
\item łagodna chromatyna,
\item normalne jądra komórki,
\item mitoza.
\end{itemize}
Atrybuty te przyjmują wartości od 1 do 10 i są one wybierane przy pomocy suwaków.

Po wybraniu przycisku "Oblicz", następuje wyznaczenie prawdopodobieństwa występowania raka piersi, a informacje wynikowe zostają wyświetlone w postaci wartości procentowej i schematu drzewa decyzyjnego, które doprowadziło do tego wyniku.

\end{document}